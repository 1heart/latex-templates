\documentclass[12pt]{article}
\usepackage[margin=0.5in]{geometry}
\usepackage{amsmath, amsthm, amsfonts, tikz, algpseudocode}
\usepackage[plain]{algorithm}
\usepackage[framemethod=default]{mdframed}

\theoremstyle{plain}
\newtheorem*{theorem}{Theorem}
\newtheorem*{lemma}{Lemma}
\newtheorem*{claim}{Claim}
\newtheorem*{definition}{Definition}
\newtheorem*{corollary}{Corollary}

%%%% TITLE

\title{Title}
\date{Date}
\author{Yixin Lin}

\begin{document}
\maketitle
% \abstract{Abstract text.}
\begin{mdframed}
\tableofcontents
\end{mdframed}
\newpage

%%%% BODY %%%%

\part*{Part 1.}

%%%% UTILS

% \begin{theorem}
% This is a theorem statement.
% \end{theorem}

% \proof{
% This is a proof.
% }
% $\qedsymbol$



% This is how you split an equation:

% \[
%     \begin{split}
%         n^2 + n + 1 &=
%         \\
%         &\leq n^2 + n^2 + n^2
%         \\
%         &= 3n^2
%         \\
%         &\leq c \cdot 2n^3
%     \end{split}
% \]



% This is a table:

% \begin{table}[ht]
%     \centering
%     \begin{tabular}{c || c | c | c | c | c}
%         & \(x \mod 5 = 0\)
%         & \(x \mod 5 = 1\)
%         & \(x \mod 5 = 2\)
%         & \(x \mod 5 = 3\)
%         & \(x \mod 5 = 4\)
%         \\
%         \hline
        
%         \(x0\) & 0 & 2 & 4 & 1 & 3 \\
%         \(x1\) & 1 & 3 & 0 & 2 & 4 \\
%     \end{tabular}
% \end{table}



% \begin{algorithm}[]
%     \begin{algorithmic}[1]
%         \Function{Quick-Sort}{$list, start, end$}
%             \If{$start \geq end$}
%                 \State{} \Return{}
%             \EndIf{}
%             \State{} $mid \gets \Call{Partition}{list, start, end}$
%             \State{} \Call{Quick-Sort}{$list, start, mid - 1$}
%             \State{} \Call{Quick-Sort}{$list, mid + 1, end$}
%         \EndFunction{}
%     \end{algorithmic}
%     \caption{Start of QuickSort}
% \end{algorithm}

\end{document}
